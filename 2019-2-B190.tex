% !Mode:: "TeX:UTF-8"
%!TEX program  = xelatex

\documentclass[bwprint]{cumcmthesis} 
\usepackage{amssymb}
\usepackage{amsmath}
\usepackage{listings}
\usepackage{graphicx}
\lstset{
    columns=fixed,       
    numbers=left,                                        % 在左侧显示行号
    frame=none,                                          % 不显示背景边框
    basicstyle=\small,
    backgroundcolor=\color[RGB]{245,245,244},            % 设定背景颜色
    keywordstyle=\color[RGB]{40,40,255},                 % 设定关键字颜色
    numberstyle=\footnotesize\color{darkgray},           % 设定行号格式
    commentstyle=\it\color[RGB]{0,96,96},                % 设置代码注释的格式
    stringstyle=\rmfamily\slshape\color[RGB]{128,0,0},   % 设置字符串格式
    showstringspaces=false,                              % 不显示字符串中的空格
    language=matlab,                                        % 设置语言
}

\title{大巴车的速度估计问题}

\begin{document}
    \maketitle
    \begin{abstract}
    本文首先采用将速度函数视为信号的思想,将题目中所给的数据看做采样点,并为便于数据处理对原始采样点进行三次样条插值。之后,应用快速傅里叶变换,找出信号中的主要频率部分,并以此为基础计算出了速度函数的解析表达式。

    其次,应用之前求出的速度函数的表达式,验证三位司机是否超速。结果表明三位司机均有超速行为。

    最后,应用熵权法求出影响司机驾驶水平的一些因素的客观权重,再应用层次分析法确立主观权重,将二者结合,应用模糊综合评价法的方法,最终给出甲、乙、丙三位司机驾驶水平的评分,分别为$93.58$分、$64.29$分和$80.23$分,结果表明甲司机的驾驶水平最高。
    \keywords{三次样条插值\quad 快速傅里叶变换\quad 熵权法\quad 层次分析法\quad 模糊综合评价法}
    \end{abstract}
    \section{问题的背景与重述}
        \subsection{引言}
        大巴车交通事故通常将造成比较严重的上伤亡后果。欲避免事故的发生,关键是要有一套行之有效、处罚严厉的管理制度$^{\cite{bib:one}}$。为了使大巴车能够安全运营并有效处理紧急事件,某大巴车出租公司对正在运营的大巴车进行监测。行车记录仪可以每隔一段时间记录从事运营的车辆实际行驶的里程数,若能由此推算出大巴车的任意时刻的速度,并对司机的驾驶水平加以评价与管理,则可达到大大降低运营风险的目的。因此,对该问题的研究具有重要的实际意义。
        \subsection{问题的提出}
        已知$0$-$15$小时内每隔半小时甲、乙、丙三位司机的行驶路程,以及高速公路的限速,建模研究以下问题:
        \begin{enumerate}
            \item 求解三辆大巴车各自运行的速度函数;
            \item 判断三位司机是否有超速行为;
            \item 评价三位司机的驾驶水平。
        \end{enumerate}
    \section{问题分析}
        \subsection{问题一分析}
        问题一要求给出三辆大巴车各自运行的速度函数。注意到题目中所提供的数据为离散取值的若干个数据点,而问题则要求给出一个连续的函数,通常这需要对数据进行拟合等处理来完成。然而,受到信号压缩与重构这一过程的启发,本文采用离散傅里叶变换(DFT)的方法,在已知数据的基础上利用MATLAB软件重构出各车的速度函数。

        首先,将相邻时刻间大巴车行驶的平均速度近似为中间时刻的瞬时速度。其次,利用三次样条插值将数据补充为等时间间隔的点。最后利用快速傅里叶变换(FFT)算法处理数据,在舍去幅度值较低的频率后,重构出所求的速度函数,并检验该函数与原始数据的吻合度。

        \subsection{问题二分析}
        问题二要求判断三位司机是否有超速行为。根据问题一所求出的结果,即可判断函数是否存在某一点使得该时刻的速度大于高速公路的限速。更进一步,还可根据速度函数求出各司机处于超速状态下的总时长。
        \subsection{问题三分析}
        问题三要求评价三位司机的驾驶水平,此题需要建立一个评价模型。首先,选取评价司机的指标。本文选取了司机驾驶的平均速度作为整体速度的衡量;选取每隔半小时行驶里程的样本变异系数衡量数据的分散程度;并选取司机处于超速状态下的时间长度作为司机驾驶的安全程度。在此基础上对司机的驾驶水平进行评价。
        
        其次,选择评价模型。常用的评价模型有层次分析法、灰色关联度分析以及模糊综合评价法等,考虑到层次分析法在构造判断矩阵时受专家的主观影响较大,导致最终评估结果不符合实际$^{\cite{bib:two}}$,而灰色关联度分析则要求样本数据具有时间序列的特性,故本文采用了模糊综合评价法,分别通过熵权法与层次分析法确定三个指标的客观权重与主观权重,进而确定组合权重。再确定评语集,构造模糊关系矩阵,最终对三个司机的驾驶水平给出评价。
    \section{模型的假设}
        \begin{enumerate}
            \item 大巴车在高速公路上行驶,可以不用考虑停车情况;
            \item 大巴车的速度函数为定义在$[0,15]$上的一阶连续可导函数;
            \item 所测得的数据准确无误。
        \end{enumerate}
    \section{主要符号说明}
        \begin{center}
            \begin{tabular}{cc}
                \hline
                符号	&  意义 \\ \hline
                $F_n$ & 第$n$个点代表的频率 \\ 
                $a_n$ & 第$n$个复数的实部 \\ 
                $b_n$ & 第$n$个复数的虚部 \\ 
                $A_n$ & 第$n$个点的幅度特性 \\ 
                $\widehat{A}_n$ & 第$n$个点的幅度 \\ 
                $v^*(t)$ & $[0,1]$上的信号 \\ 
                $v(t)$ & $[0,15]$上的速度函数 \\
                \hline
            \end{tabular}
        \end{center}
    \section{模型的原理}
        \subsection{离散傅里叶变换(Discrete Fourier Transform, DFT)}
        在被采样信号和周期信号测量处理中,离散傅里叶变换得到广泛的应用$^{\cite{bib:three}}$。离散傅里叶变换是傅里叶变换的离散形式,即在时域和频域上都呈离散的取值。在形式上,时域和频域上变换两端的序列是有限长的,而实际上这两组序列都被认为是离散的经周期延拓后的信号的主值序列。设已有数据列为$x(0),x(1),\dots,x(N-1)$,则DFT的公式为:
        \begin{equation}
            \label{eq:DFT}
            X(j) = \sum_{n=0}^{N-1}x(n)e^{-2\pi i\frac{nj}{N}}
        \end{equation}
        
        在本文中,采用快速傅里叶变换(Fast Fourier Transform, FFT)算法进行计算。该算法利用了DFT中的奇偶性与虚实性极大地简化了计算,将DFT中$\mathcal{O}(N^2)$的计算量减小至FFT的$\mathcal{O}(NLog2N)$,使得计算机得运算时间大大缩短。

        \subsection{熵权法(Entropy Weight Method, EWM)}
        熵权法是一种根据各项指标观测值所提供的信息量的大小来确定指标权重的方法$^{\cite{bib:four}}$。若某个指标的信息熵越小,表明指标值得变异程度越大,提供的信息量越多,在综合评价中所能起到的作用也越大,其权重也就越大。因此,可利用信息熵这个工具,计算出各个指标的权重,熵权法也由此而来。其计算步骤如下:
        \begin{enumerate}
            \item 数据标准化;
            
            假设给定了k个指标$X_1,X_2,\dots,X_k$,其中$X_i = ({x_1,x_2,\dots,x_n})^T$。假设对各指标数据标准化后的值为$Y_1,Y_2,\dots,Y_k$,那么
            \[
                Y_{ij}=
            \begin{cases}
                \label{eq:EWM1}
                \frac{X_{ij}-\min(X_i)}{\max(X_i)-\min(X_i)}  & \text{若}X\text{的值越大,则}X\text{越好} \\
                \frac{\max(X_i)-X_{ij}}{\max(X_i)-\min(X_i)}  &\text{若}X\text{的值越小,则}X\text{越好}  
            \end{cases}
            \]

            \item 求个指标得信息熵;
            
            根据信息论中信息熵的定义,一组数据的信息熵为
            \begin{equation}
                \label{eq:EWM2}
                E_j = -\log(n)^{-1}\sum_{i=1}^{n}p_{ij}\log p_{ij}
            \end{equation}
            其中$p_{ij} = \frac{Y_{ij}}{\sum_{i=1}^{n}Y_{ij}}$.
            \item 确定各指标的权重。
            
            根据信息熵的计算公式,计算出各个指标的信息熵为$E_1,E_2,\dots,E_k$。通过信息熵计算各指标的权重:
            \begin{equation}
                \label{eq:EWM3}
                W_i = \frac{1-E_i}{k-\sum E_i}\quad (i = 1,2,\dots,k)
            \end{equation}
        \end{enumerate}
        
        \subsection{层次分析法(Analytic Hierarchy Process, AHP)}
        层次分析法是确定指标主观权重最常用的方法,其主要思想是以评价目标为出发点,对各评价指标进行分解,构造出指标之间的层级关系,再结合判断矩阵计算权重系数$^{\cite{bib:five}}$。其基本步骤如下:
        \begin{enumerate}
            \item 构造判断矩阵;
            
            假设要比较某一层$n$个因素$C_1,C_2,\dots,C_n$对上层一个因素$O$的影响,每次取两个因素$C_i$和$C_j$,用$a_{ij}$表示$C_i$和$C_j$对$O$的影响之比,全部比较结果可用成对比较矩阵表示:
            \begin{equation}
                \label{eq:AHP1}
                A = (a_{ij})_{n\times n},\quad a_{ij}>0,\quad a_{ji} = \frac{1}{a_{ij}}
            \end{equation}
            \item 确定各指标权重;
            
            对于成对比较矩阵$A$,找出其最大特征值$\lambda$对应的特征向量归一化后作为权向量$w$,即$w$满足:
            \begin{equation}
                \label{eq:AHP2}
                Aw=\lambda w,\quad \|w\|_2 = 1
            \end{equation}
            \item 一致性检验。
        
            最后,根据成对比较矩阵$A$的一些数字特征进行一致性检验。检验方法如下:
            \begin{equation}
                \label{eq:AHP3}
                CR = \frac{CI}{RI}
            \end{equation}
            其中$CI$称为一致性指标,定义式为$CI=\frac{\lambda-n}{n-1}$,而$RI$是随机一致性指标,与$A$的阶数$n$有关。当$CR<0.1$时,则称通过了一致性检验,此时$w$可作为权向量。
        \end{enumerate}

        \subsection{模糊综合评价法}
        模糊综合评价是一种基于模糊数学的评价方法,根据模糊数学的隶属度理论把定性评价转化为定量评价,能较好地解决模糊的、难以量化的问题$^{\cite{bib:six}}$。本文按如下步骤求解该模型:
        \begin{enumerate}
            \item 建立评语集并划分评价等级;
            \item 计算指标对各评语的隶属度,进而构造模糊评价矩阵;
            \item 根据先前确定的权重,计算模糊评价向量。
        \end{enumerate}
    \section{模型的求解}
        \subsection{数据的预处理}
        题目要求给出大巴车的速度函数。为将问题简化,首先将题目中的原始数据进行预处理。将相邻半小时内大巴车行驶的平均速度近似为大巴车在中间时刻的平均速度,这样,我们得到$t=0.25, 0.75, 1.25,\dots, 14.75h$共$30$个时刻大巴车的瞬时速度,如图1所示:
        \begin{figure}[htbp]
            \centering
            \includegraphics[width=1\linewidth]{rawdata.png}
            \caption{瞬时速度}
        \end{figure}
            
        为方便后续的数据处理,采用三次样条插值的方法,对速度函数$v(t)$进行插值,得到插值后$t=0, 0.5, 1,\dots, 30h$时刻的瞬时速度值。这样我们得到一系列等时间间隔的瞬时速度的数据点,如图2所示:
        \begin{figure}[htbp]
            \centering
            \includegraphics[width=1\linewidth]{interpolate.png}
            \caption{三次样条插值后的数据点}
        \end{figure}
        
        \subsection{基于FFT的速度求解}
        经过对速度函数$v(t)$的采样,我们即可利用FFT重构出完整的速度函数。下文将$v(t)$假想为定义域为$[0,1]$的信号$v^*(t)$进行求解,最后只需对时间轴进行拉伸变换即可得到所求的函数。现已将信号以$F_s=60\mathrm{Hz}$的采样频率进行采样,采样点数$N=60$,故FFT所得结果的第$n$个点代表的频率为$F_n=\frac{(n-1)F_s}{N}$。在MATLAB中,利用FFT算法对该序列进行离散傅里叶变换(具体代码见附录),并对得到的$N$个复数$a_n+b_ni$取模,得到各频率值下的幅度特性$A_n$,如图3所示:
        \begin{figure}[htbp]
            \centering
            \includegraphics[width=1\linewidth]{mod.png}
            \caption{幅度特性$A_n$}
        \end{figure}

        通过如上结果,我们进一步得到$n=1$处点的信号直流分量$\frac{A_1}{N}$,以及$2\leq n\leq \frac{N}{2}$时,$n\mathrm{Hz}$信号的幅度$\frac{2A_n}{N}$,且其相位为$\mathrm{atan2}(b_n,a_n)$,其中$\mathrm{atan2}$为四象限反正切函数。

        故我们得到: 
        \begin{equation*}
            v^*(t)=XXX
        \end{equation*}

        最终得到的速度函数可以写成如下形式:
        \begin{equation*}
            v(t)=XXX
        \end{equation*}
        其图像大致如图4所示:
        \begin{figure}[htbp]
            \centering

            \includegraphics[width=1\linewidth]{v.png}
            \caption{速度函数}
        \end{figure}

        \subsection{模型的检验}
            现在,对所求出的速度函数与原始数据的吻合度进行检验。分别对相邻半小时下的速度函数进行积分,求出相邻时间里大巴车行驶路程的模拟值,并与实际值加以比较。具体数值如图5所示:
            \begin{figure}[htbp]
                \centering
                \includegraphics[width=1\linewidth]{compare.png}
                \caption{模拟值与实际值的比较}
            \end{figure}

            经计算,发现每一点处的相对误差都很小(小于$3\%$),在可接受的范围之内,因此,本文建立的模型是有效可靠的。
        
        \subsection{超速行为的判定}
        将问题一中求得的速度函数与高速公路的限速画入同一张图像中:
        \begin{figure}[htbp]
            \centering
            \includegraphics[width=1\linewidth]{v100.png}
            \caption{速度函数与限速}
        \end{figure}

        可清晰地看出甲、乙、丙三位司机均存在超速的行为。更进一步地,可以求解得出三位司机超速的时间分别为$0.990$小时、$4.455$小时与$2.550$小时。

        \subsection{司机驾驶水平的评价}
        考虑到大巴车出租公司运营中对车辆行驶快速、平稳和安全的实际需求,本文分别选取平均速度、每半小时行驶里程的样本变异系数和超速时间作为衡量三项要求的指标。其中,样本变异系数的计算公式为:
        \begin{equation*}
            \nu = \frac{\sqrt{DX}}{EX} = \frac{\sqrt{\frac{1}{n}\sum_{i=1}^{n}\left(X_i-\bar{X}\right)^2}}{\frac{1}{n}\sum_{i=1}^{n}X_i}
        \end{equation*}

        由此,我们得到三位司机驾驶个各项指标如下表:
        
        \begin{table}[htbp]
            \centering
            \caption{驾驶指标}
                \begin{tabular}{|c|c|c|c|}
                \hline
                & 平均速度 & 变异系数 & 超速时间 \\
                \hline
                甲     & 90.77 & 0.086 & 0.990 \\
                \hline
                乙     & 92.04 & 0.112 & 4.455 \\
                \hline
                丙     & 92.66 & 0.110 & 2.250 \\
                \hline
                \end{tabular}
            \label{tab:seven}
          \end{table}

        接下来,对数据进行标准化处理。考虑到大巴车的平均速度应尽可能地快,而变异系数以及超速时间应越小越好,故对两种情况分别采用两种标准化的方法,使得标准化系数均在$[0,1]$之间,且越接近$1$代表司机的该项指标越为优秀。

            \subsubsection{基于EWM的客观权重确定}
            通过EWM确定三项指标分别所占的客观权重。在标准化的基础上,我们利用以下公式(\ref{eq:EWM2})(\ref{eq:EWM3})求出各指标的信息熵为$0.613$、$0.233$以及$0.608$;且各指标的权重为$0.250$、$0.496$和$0.253$。


            \subsubsection{基于AHP的主观权重确定}
            考虑到大巴车在高速上长期行驶,故安全性应明显重要于其他两因素。此外,高速上以平稳的速度行驶也比行驶的速度更快重要,因此利用$1$-$9$尺度,根据(\ref{eq:AHP1})(\ref{eq:AHP2})(\ref{eq:AHP3}),构造如下成对比较矩阵:
            \[
            \begin{pmatrix}
                1 & \frac{1}{3} & \frac{1}{9}  \\
                3 & 1 & \frac{1}{5}  \\
                9 & 5 & 1  \\
            \end{pmatrix}
            \]
            经计算,该矩阵的特征向量为$(0.395,1.000,4.217)^T$,经过归一化得到三项指标的主观权重分别为$0.070$、$0.178$和$0.751$。

            一致性检验显示,该矩阵唯一的实特征值为$3.029$,故$CI=\frac{\lambda-n}{n-1}=0.015$。经查表得当$n=3$时$RI=0.58$,故$CR=\frac{CI}{RI}=0.025<0.1$,即该矩阵通过了一致性检验。

            综上,得到了三个指标的客观权重与主观权重,取其平均值即得最终的组合权重$R$如下表所示。
            \begin{table}[htbp]
                \centering
                \caption{组合权重$R$}
                    \begin{tabular}{|c|c|c|c|}
                    \hline
                    & 平均速度 & 变异系数 & 超速时间 \\
                    \hline
                    客观权重  & 0.250 & 0.496 & 0.253 \\
                    \hline
                    主观权重  & 0.070 & 0.178 & 0.751 \\
                    \hline
                    组合权重  & 0.160 & 0.337 & 0.502 \\
                    \hline
                    \end{tabular}
                \label{tab:eight}
              \end{table}
            \subsubsection{基于模糊综合评价法的驾驶水平评价}
            在各项指标经过标准化处理后,所有的指标取值都在$[0,1]$上。依照此数据特性,本文建立的评语集为$\{\text{较差,一般,较好,很好,非常好}\}$,其评价等级的划分标准如下表所示。
            \begin{table}[htbp]
                \centering
                \caption{评价等级划分标准}
                    \begin{tabular}{|c|c|c|c|c|c|}
                    \hline
                    评价等级  & 较差    & 一般    & 较好    & 很好    & 非常好 \\
                    \hline
                    标准分值  & 0.00     & 0.25  & 0.50   & 0.75  & 1.00 \\
                    \hline
                    \end{tabular}
                \label{tab:nine}
            \end{table}
            
            依照此划分标准,可以计算各司机各项指标对各评语的隶属度,进而构造出各司机的模糊评价矩阵。令$A_1$,$A_2$,$A_3$分别代表甲、乙、丙的评价矩阵,那么:
            \[
                A_1 = 
                \begin{pmatrix}
                    1 & 0 & 0 & 0 & 0  \\
                    0 & 0 & 0 & 0 & 0  \\
                    0 & 0 & 0 & 0 & 1  \\
                \end{pmatrix}
            \]
            \[    
                A_2 = 
                \begin{pmatrix}
                    0 & 0 & 0.326 & 0.674 & 0  \\
                    0 & 0 & 0 & 0 & 0  \\
                    1 & 0 & 0 & 0 & 0  \\
                \end{pmatrix}
            \]
            \[
                A_3 = 
                \begin{pmatrix}
                    0 & 0 & 0 & 0 & 1  \\
                    0.695 & 0.305 & 0 & 0 & 0  \\
                    0 & 0 & 0.455 & 0.545 & 0  \\
                \end{pmatrix}
            \]
            在已知权重向量$R$的基础上,利用模糊变换$B_i=RA_i$,求出甲、乙、丙三位司机的模糊综合评价向量。再通过下述评分等级百分制赋分(如下表),对三位司机的驾驶水平进行比较。
            \begin{table}[htbp]
                \centering
                \caption{评价等级百分制赋分}
                    \begin{tabular}{|c|c|c|c|c|c|}
                    \hline
                    评价等级  & 较差    & 一般    & 较好    & 很好    & 非常好 \\
                    \hline
                    百分制赋分 & 60    & 70    & 80    & 90    & 100 \\
                    \hline
                    \end{tabular}
                \label{tab:ten}
            \end{table}
            
            最终,我们得到甲、乙、丙三位司机的最终评分分别为$93.58$分、$64.29$分和$80.23$分。即甲司机的驾驶水平最好,其次是丙司机,最后是乙司机。
            \begin{table}[htbp]
                \centering
                \caption{最终排名}
                    \begin{tabular}{|c|c|c|c|c|c|c|c|}
                    \hline
                            & \multicolumn{5}{c|}{综合评价向量}    & 得分    & 排名 \\
                    \hline
                    甲     & \multicolumn{5}{c|}{(0.160,\quad0,\quad0,\quad0,\quad0.840)}                 & 93.58 & 1 \\
                    \hline
                    乙     & \multicolumn{5}{c|}{(0.840,\quad0,\quad    0.052,\quad0.108,\quad0)}                 & 64.29 & 3 \\
                    \hline
                    丙     & \multicolumn{5}{c|}{(0.234,\quad0.103,\quad0.228,\quad0.274,\quad0.160)}                 & 80.23 & 2 \\
                    \hline
                    \end{tabular}
                \label{tab:addlabel}
            \end{table}
    \section{模型的评价与改进}
    本文有如下优点:其一,利用DFT重构速度函数,相比于拟合数据等其他方法更具有说力,且利用FFT算法进行求解,缩短了计算机的运算时间;其二,在确定各指标的权重时,采用EWM-AHP的组合权重模型,是主观与客观有机结合的方法,既避免了人过度受主观因素的影响,又能结合实际情况而不受表面数据的拘泥;其三,在评价模型的选择上,模糊综合评价法能通过精确地数字手段处理模糊的评价对象,能对蕴藏信息呈现模糊性的资料作出比较科学、合理、贴切实际的量化评价。

    然而,本文还存在着些许不足。在速度的估计上,利用平均速度估计某一点的瞬时速度虽然有一定的可靠性,但无法保证模型进一步的精确度。同时,在对序列做DFT时,该序列将被视为周期延拓后的序列,使得在对$t=0$以及$t=15$附近的速度估计不够准确。

    未解决上述问题,则需更精确的模型或数据处理手段。再此提出几个接下来需要深入研究的方向:一是加强插值方法的合理性,增加插值点的个数,从而可以对FFT之后的频谱进行进一步的分析;二是改进模型,引入小波分析的知识,利用小波变换的优越性弥补傅里叶变换的不足,从而达到裨补阙漏的效果。
    
    \newpage
    \begin{thebibliography}{9}
        \bibitem{bib:one} 佚名.国外如何对长途客车进行安全管理[Z]. : ,2019.
        \bibitem{bib:two} 齐天乐 胡惠军.基于AHP-熵权综合评价法的装备维修保障质量影响因素评估[J].价值工程,:,2019
        \bibitem{bib:three} 陈岗.离散数列的傅里叶变换[J].科技资讯,:,2016.09(c):141-142.
        \bibitem{bib:four} 蔡文军,李晓松.海军舰船装备保障能力评估理论与方法[M]
        \bibitem{bib:five} 刘吉成 闫文婧.光伏企业价值创造能力评价研究[J].会计之友,:,2019.3():78-82.
        \bibitem{bib:six} 
        张翔宇,宋瑞明,刘姝芳,张凤燃,谷晓伟,程春晓.模糊综合评价模型在水生态文明评价中的应用[J].水力发电,:,2019.45(2):39-42.
       
    \end{thebibliography}
    \newpage
    \appendix
        \section{matlab源码}
        \begin{lstlisting}[language=matlab]
%%% main.m
N = 1000;
ddriver = 1;
data = importdata('data');
time = data(:, 1) - 0.25;
data1 = data(:, ddriver+1);
vlist = data1 .* 2;
vlist = vlist';
vtime = linspace(0, 15, 61);
interMethod = 'spline';
vInter = interp1(time, vlist, vtime, interMethod, 'extrap');

f = getFFTfun(vInter, 0, 15, 0.0);
x = linspace(0, 15, N);
y = arrayfun(f, x);

figure
s(1) = subplot(221);
plot(time, vlist, 'r.', x, y, 'b', 0:15, ones(1,16)*100, 'g');

s(2) = subplot(222);
I = arrayfun(@(v)quad(f, v-0.5, v), data(:, 1));
plot(data(:, 1), data1, 'r.');
hold on;
plot(data(:, 1), I, 'g.');

s(3) = subplot(223);
plot(data(:, 1), I-data1, 'b.');

s(4) = subplot(224);
plot(time, vlist, 'r+');
hold on;
plot(vtime, vInter, 'b.');

title(s(1), sprintf('result of driver%d', ddriver), 'FontSize', 15);
title(s(2), 'compare the integrate value', 'FontSize', 15);
title(s(3), 'Error of the integrate', 'FontSize', 15);
title(s(4), sprintf('result of interpolate, method:%s', interMethod), 'FontSize', 15);


%%% getFFTfun.m
function fun = getFFTfun(xlist, tmin, tmax, threshold)
    N = length(xlist) - 1;
    xfft = fft(xlist, N);
    xMod = abs(xfft);
    xMod = xMod / (N/2);
    xMod(1) = xMod(1) / 2;
    xarg = arrayfun(@(x)atan2(imag(x), real(x)), xfft(1:floor(N/2)));
    function xmean = ffun(t)
        tStd = (t-tmin)/(tmax-tmin);
        xmean = xMod(1);
        for i = 2: length(xarg)
            if xMod(i) <= threshold
                continue;
            end
            xmean = xmean + xMod(i)*cos(2*(i-1)*pi*tStd+xarg(i));
        end
    end
    fun = @ffun;
end

        \end{lstlisting}

\end{document} 